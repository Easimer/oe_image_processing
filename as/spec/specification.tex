\documentclass[10pt, draft]{article}

\usepackage[margin=1.0in]{geometry}
\usepackage[utf8]{inputenc}
\usepackage{csquotes}
\usepackage{t1enc}
\usepackage[hungarian]{babel}
%\usepackage{algorithm}
%\usepackage{algorithmic}
\usepackage{changepage}
\usepackage{amsmath}
\usepackage{amssymb}
\usepackage{xcolor}
\usepackage[
    backend=biber,
    style=ieee
]{biblatex}
\usepackage[pdfusetitle]{hyperref}
\usepackage{graphicx}
\graphicspath{ {./img/} }
\hypersetup{pdftex,colorlinks=true,allcolors=black}
\addbibresource{cite.bib}

\begin{document}
\title{Képfeldolgozás alap algoritmusai önálló feladat specifikáció}
\author{Mészáros Dániel}
\date{2020/21/1}
\maketitle

A féléves feladatom egy olyan program készítése, amely beégetett feliratokkal
rendelkező videókat dolgoz fel: először megkeresi és lemaszkolja a képen
található feliratokat, majd a keletkezett űrt kitölti egy image inpainting
algoritmus segítségével.

A program kimenete ezen mesterségesen kiegészített képek sorozata.

\section{Felirat érzékelése és körbehatárolása}
A felirat érzékelési algoritmus a 2007-ben Huang, et al.
\cite{huang2007intelligent} által bemutatott TV reklámblokk detektor, ami
feliratok jelenléte alapján határozza meg, hogy a videóban most épp reklám megy
vagy sem \footnote{a szerző szerint Taiwan-ban minden TV műsor feliratozva van,
kivéve a reklámok}.

Az algoritmus $Y C_r C_b$ színtérben operál, mert így kevésbé érzékenyebb a
képnek az emberi látás számára irreleváns változásokra.

% TODO:
Felirat BB előállítása: edge detection, 3-átlagolás (NTSC, 480i, frame
rate-függetlenítés), gate filter, BB kiszámítása

Miután megvan a BB: vesszük a szövegdoboz hisztogramját a $C_b$ és $C_r$
csatornákban és azokat a pixeleket, amelyek a leggyakoribb színeket veszik fel
és megjelöljük őket a maszkban.
Mivel mi a feliratot szeretnénk eltávolítani, ezért a szöveghez tartozó
pixelek a 0, a többi pixelek pedig az 1 értéket fogják felvenni.
\cite{huang2007intelligent}

\section{Image inpainting}

Miután eltávolítottuk a feliratokat a képkockából, be kell töltenünk az
keletkezett üres részt.
Ehhez Bertalmio, et al. \cite{bertalmio2000image} inpainting technikáját
használom.
Az algoritmus lényege, hogy ha adott valamilyen képterület, illetve ezen
területnek a határa, akkor ahhoz, hogy kitöltsük az űrt, ahhoz az űrön kívül
levő "információt" kell bepropagálni a határon keresztül a területbe.
Ezzel azt érjük el, hogy az izofóta\footnote{azonos fényerejű pixeleket
összekötő vonal} vonalakat meghosszabbítjuk.
Ezt az algoritmus iteratívan viszi végbe, folyamatosan feltöltve az üres
területet.

Szeretnénk megtudni, hogy ahhoz, hogy megkapjuk egy adott pixel következő állapotát, mennyit kell hozzáadni a jelenlegi intenzitáshoz, azaz ha

\[ I^{n+1}(i, j) = I^n(i, j) + \Delta t I^{n}_{t} (i,j) \]

akkor mennyi a $I^{n}_t (i,j)$.

Ehhez először kiszámoljuk az $L^n (i, j)$-t, ami a jelenlegi kép Laplace-szűrő alkalmazása után.
Ebből meghatározhatjuk az információ változásának mennyiségét:
\[ \overrightarrow{\delta L^{n}} = (L^n(i+1,j) - L^n(i-1,j), L^n(i,j+1) - L^n(i,j-1)) \]
A vektor komponensei lényegében a Laplace-olt képen az X illetve Y-tengelyeken mért változás.

Szükséges még továbbá a propagációs irány $N(i, j)$, amelyet a kép grádienséből számolhatunk ki:
\[ N^{n}(i, j) = (-I^{n}_y(i, j), -I^{n}_x(i, j)) \]
Ezt a vektort a hosszával elosztva normalizáljuk.

Az utolsó két érték amit ki kell számolni azelőtt, hogy megkaphatnánk $I^{n}_t (i,j)$-t,
az a $ \beta^n $ és a $ | \nabla I^n(i, j) | $.

Az előbbi a $ \overrightarrow{ \delta L^n } $ és a normál vektor skalárszorzata:
\[ \beta^{n} = \overrightarrow{ \delta L^n }(i, j) \cdot \frac{ N^n (i, j) }{ N^n (i, j) } \]
míg az utóbbi:
\[ | \nabla I^n(i, j) | =
\begin{cases}
    \sqrt{(I^n_{x b m})^2 + (I^n_{x f M})^2 + (I^n_{y b m})^2 + (I^n_{y b M})^2} &\mbox{ha } \beta^n > 0 \\
    \sqrt{(I^n_{x b M})^2 + (I^n_{x f m})^2 + (I^n_{y b M})^2 + (I^n_{y b m})^2} &\mbox{ha } \beta^n < 0
\end{cases}
\]
Alsóindexben az x/y a parciális derivált változója, a b/f pedig a különbségek iránya (b - negatív irány, f - pozitív irány).
Az m/M azt jelzi, hogy a különbségeknek a nullával való minimumát vagy maximumát vesszük.
Például a $I^n_{y b M}$-et definiálhatjuk úgy, hogy $ max(0, I^n(i, j - 1) - I^n(i, j + 1)) $,
vagy $I^n_{x f m} = min(0, I^n(i + 1, j) - I^n(i - 1, j)$

% TODO: 5. számú egyenletet most már össze tudjuk rakni
% TODO: iterációs ciklus, anizotrópikus diffúzió
% TODO: színtér
% TODO: piramis

\section{Felhasznált könyvtárak}

\printbibliography
\end{document}

\documentclass[10pt, draft]{article}

\usepackage[margin=1.0in]{geometry}
\usepackage[utf8]{inputenc}
\usepackage{csquotes}
\usepackage{t1enc}
\usepackage[hungarian]{babel}
%\usepackage{algorithm}
%\usepackage{algorithmic}
\usepackage{changepage}
\usepackage{amsmath}
\usepackage{amssymb}
\usepackage{xcolor}
\usepackage[
    backend=biber,
    style=ieee
]{biblatex}
\usepackage[pdfusetitle]{hyperref}
\usepackage{graphicx}
\graphicspath{ {./img/} }
\hypersetup{pdftex,colorlinks=true,allcolors=black}
\addbibresource{cite.bib}

\begin{document}
\title{Képfeldolgozás alap algoritmusai önálló feladat specifikáció}
\author{Mészáros Dániel}
\date{2020/21/1}
\maketitle

A féléves feladatom egy olyan program készítése, amely beégetett feliratokkal
rendelkező videókat dolgoz fel: először megkeresi és lemaszkolja a képen
található feliratokat, majd a keletkezett űrt kitölti egy image inpainting
algoritmus segítségével.

A program kimenete ezen mesterségesen kiegészített képek sorozata.

\section{Felirat érzékelése és körbehatárolása}
A felirat érzékelési algoritmus a 2007-ben Huang, et al.
\cite{huang2007intelligent} által bemutatott TV reklámblokk detektor, ami
feliratok jelenléte alapján határozza meg, hogy a videóban most épp reklám megy
vagy sem \footnote{a szerző szerint Taiwan-ban minden TV műsor feliratozva van,
kivéve a reklámok}.

Az algoritmus $Y C_r C_b$ színtérben operál, mert így kevésbé érzékenyebb a
képnek az emberi látás számára irreleváns változásokra.

% TODO:
Felirat BB előállítása: edge detection, 3-átlagolás (NTSC, 480i, frame
rate-függetlenítés), gate filter, BB kiszámítása

Miután megvan a BB: vesszük a szövegdoboz hisztogramját a $C_b$ és $C_r$
csatornákban és azokat a pixeleket, amelyek a leggyakoribb színeket veszik fel
és megjelöljük őket a maszkban.
Mivel mi a feliratot szeretnénk eltávolítani, ezért a szöveghez tartozó
pixelek a 0, a többi pixelek pedig az 1 értéket fogják felvenni.
\cite{huang2007intelligent}

\section{Image inpainting}

Miután eltávolítottuk a feliratokat a képkockából, be kell töltenünk az
keletkezett üres részt.
Ehhez Bertalmio, et al. \cite{bertalmio2000image} inpainting technikáját
használom.
Az algoritmus lényege, hogy ha adott valamilyen képterület, illetve ezen
területnek a határa, akkor ahhoz, hogy kitöltsük az űrt, ahhoz az űrön kívül
levő "információt" kell bepropagálni a határon keresztül a területbe.
Ezzel azt érjük el, hogy az izofóta\footnote{azonos fényerejű pixeleket
összekötő vonal} vonalakat meghosszabbítjuk.
Ezt az algoritmus iteratívan viszi végbe, folyamatosan zsugorítva az üres
területet.

% TODO: módszer leírása, implementációja

\section{Felhasznált könyvtárak}

\printbibliography
\end{document}
